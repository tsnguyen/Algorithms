\documentclass{article}
\usepackage{amsmath}
\usepackage{mathtools}

\title{Message integrity}

\begin{document}
\maketitle

\section{Message Authorization Code (MAC)}

\subsection{MAC and CRC}
The goal is to guarantee integrity of the message, with no concern of
confidentiality. Real world examples are protecting public binaries on disk
(such as system files on Windows machine can be visible for evrybody, but must
be protected against corruption from viruses), protecting banner ads on web
pages (companies don't care if people seeing their ads, but they want to make
sure nobody change the contents of the ads).

The idea to maintain the message integrity is to use message authorization codes
(MACs).

\emph{Definition: } MAC $I = (S, V)$ defined over $(\mathcal {K, M, T})$ is a
pair of algorithms:
\begin{itemize}
\item the signinng algorithm $S(k, m)$ outputs tag $t$ in $\mathcal{T}$
\item the verifying algorithm $V(k, m, t)$ outputs ``yes'' or ``no''.
\end{itemize}

Similar to encryption, MACs requires a secret key. The codes to protect the
message against random erros are cyclic redundancy check (CRC) does not require
a secret key, but it does not protect against malicious erros neither. An
attacker can easily modify message $m$ and re-compute CRC. 

\subsection{Secure MACs}
Similar to encryption, we are going to define secure MACs. We will assume for
now that the attacker's power is to perform chosen message attack. For a set of
messages $m_1, m_2, \dotsc, m_q$, attacker is given $t_i \leftarrow S(k,
m_i)$. The attacker's goal is existential forgery: to produce some new valid
message/tag pair $(m, t)$ such that $(m, t) \notin \left\lbrace (m_1, t_1),
  (m_2, t_2), \dotsc, (m_q, t_q) \right\rbrace$. The attacker then should not be
able to produce a valid tag for a new message, and given $(m, t)$ attacker
should not even produce $(m, t')$ for $t \neq t'$.

To define secure MACs, let us define a MAC game as follow: given a MAC $I = (S,
V)$ andand adversary $A$. The adversary will give the challenger a set of
messages $m_1, \dotsc, m_q$, the challenger will then produce the tags $t_1,
\dotsc, t_q$ for all the messages using the same $k \in \mathcal{K}$. The
adversary will then try to come up with an $(m, t)$ pair and give it back to the
challenger. The challenger will then ouput $b = 1$ if $V(k, m, t) = $``yes''
and $(m, t) \notin \left\lbrace (m_1, t_1), \dotsc, (m_2, t_2) \right\rbrace$,
and output $b=0$ otherwise. We define $I = (S, V)$ is a secure MAC if for all
``efficient '' adversary $A$, the advantage of $A$ over $I$, or the probability
that the challenger outputs 1, is ``negligible''.

\subsection{Example: protecting system files}
Suppose at install time, the system computes the tags for the set of system
files using the key $k$ generated from the user's password. Later a virus
infects the system and modifies system files. The user can then reboot into a
clean OS and supplies his password. A secure MAC then would be able to detect
all the modified files. Note that, however, if the virus swap the files, then
the secure MAC cannot detect this behaviour. To scope with swapping files, the
MAC can generate the tag using both the file name and the file contents.

\section{MACs based on PRFs}

Now that we know what is a secure MAC, let go ahead and construct our first
secure MAC. For a given PRF $F: \mathcal{K} \times \mathcal{X} \to \mathcal{Y}$,
we define a MAC $I_F = (S, V)$ as
\begin{itemize}
\item $S(k, m) := F(k,m)$
\item $V(k, m, t)$: ouputs ``yes'' if $t = F(k, m)$ and ``no'' otherwise.
\end{itemize}

A bad example using the construction above is to use a secure PRF $F:
\mathcal{K} \times \mathcal{X} \to \mathcal{Y}$ with $\mathcal{Y} = \lbrace 0, 1
\rbrace^{10}$. In this case, even though the PRF is secure, the $\mathcal{Y}$
space is too small. An adversary can just randomly guess the tag $t$ and still
get the advantage of $1 / 2^{10}$, which is non-negligible.

\emph{Theorem: } If $F : \mathcal{K} \times \mathcal{X} \to \mathcal{Y}$ is a
secure PRF, and $1 / | \mathcal{Y}|$ is negligible, then $I_F$ is a secure MAC.

In particular, for any efficient MAC adversary $A$ attacking $I_F$, there exists
an efficient PRF adversary $B$ attacking $F$ such that
\begin{eqnarray}
  \text{Adv}_\text{MAC} [ A, I_F] \le \text{Adv}_\text{PRF} [B, F] +
  \frac{1}{|\mathcal{Y}|} 
\end{eqnarray}
Therefore, given that $F$ is a secure PRF, $I_F$ is secure as long as
$|\mathcal{Y}|$ is large, say $|\mathcal{Y}| = 2^{80}$.

\subsection{Example}
AES can be used as a MAC for 16 byte messages . The main question is how to
conver small MAC to big MAC: how to use MAC for longer messages. There are two
main constructions used in practice:
\begin{itemize}
\item CBC-MAC: used in banking -  ANSI X9.9, X9.19, FIPS 186-3
\item HMAC: used in internet protocols: SSL, IPsec, SSH
\end{itemize}
Both of these constructions convert a small PRF ino a big PRF.

\subsection{Truncating MACs based on PRFs}
\emph{Lemma: } suppose $F: \mathcal{K} \times \mathcal{X} \to \lbrace 0, 1
\rbrace^n$ is a secure PRF, then so is $F_t(k, m) = F( k, m)[1 \dotsc t]$ for
all $1 \le t \le n$ (a truncated of a secure PRF is also a secure PRF).

Therefore, if $(S,V)$ is a MAC based on a secure PRF outputing $n$ bits, the
truncated MAC outputing $w$ bits is secure as long as $1 / 2^w$ is negligible,
say $w \ge 64$.

\section{CBC-MAC and NMAC}

As we have seen in the previous section, given a secure PRF $F$, we can
construct a secure MAC 
\begin{eqnarray}
  S(k, m) &=& F(k, m)
\end{eqnarray}
as long as $|\mathcal{Y}|$ is large. 

Our goal for this section is given a PRF for short message (such as AES),
construct a PRF for long message. From now on we let $\mathcal{X} = \lbrace 0, 1
\rbrace^n$ (i.e. n = 128 for AES).

\subsection{Construction 1: encrypted CBS-MAC (or ECBC for short)}

Let $F: \mathcal{K} \times \mathcal{X} \to \mathcal{X}$ be a PRP. Define a new
PRF $F_\text{ECBC}: \mathcal{K}^2 \times \mathcal{X}^{\le L} \to \mathcal{X}$ as
follow: first devide the message into blocks $m_1, m_2, \dotsc$ and then compute
iteratively 
\begin{eqnarray}
  t_1 &=& F(k, m_0) \\
  t_i &=& F(k, m_i \oplus t_{i-1}) \quad 1 < i \le l \\
  t &=& F(k_1, t_l)
\end{eqnarray}
which will generate the tag that has the same length as the message block.

With this construction, the message can have any length as long as it is less
than the maximum length $L$. Note that the final encryption with $k_1$ is
important: without the final encryption, the construction is called raw CBC and
is not secure.

\subsection{Construction 2: NMAC (nested MAC)}

Let $F: \mathcal{K} \times \mathcal{X} \to \mathcal{K}$ be a PRF, define a new
PRF $F_\text{NMAC}: \mathcal{K}^2 \times \mathcal{X}^{\le L} \to K$ such that:
devide the message into blocks of $n$ bits, then computes
\begin{eqnarray}
  t_0 &=& F(k, m_0) \\
  t_i &=& F(t_{i-1}, m_i) \quad 1 < i \le l\\
  t &=& F(k_1, t_l || \text{fpad})
\end{eqnarray}
which generates the tag that has the same length as the key.

Similar to the ECBC, in the NMAC construction we also devide the message into
blocks. Then the encryption goes through each block iteratively, computing new
key in each round, which will be used in the next. After going through all the
blocks of the message, the last key is paded with a fixed pad and encrypted one
more time with $k_1$ to get the final key. Without the final encryption using
$k_1$, the construction is called the cascade, and is not secure.

\subsection{Why the final encryption  using a different key is important for
  both NMAC and ECBC?}

Why the cascade is insecure? Because if we have cascade$(k, m)$ then we can
easily use $F$ to compute cascade$(k, m||w)$ for any $w$.

For ECBC, the proof is a bit more complicated. Suppose we define MAC as
$I_\text{raw} = (S,V)$ such that
\begin{eqnarray}
  S(k, m) &=& \text{rawCBC}(k, m)
\end{eqnarray}
then $I_\text{raw}$ is easily broken using a 1-chosen message attack. The
asversary works as foolows:
\begin{itemize}
\item choose an arbitrary one block message $m \in \mathcal{X}$
\item request the tag for $m$ and get $t = F(k, m)$
\item output $t$ as MAC forgery for the two block message $(m , t \oplus m)$
\end{itemize}
Indeed
\begin{eqnarray}
  \text{rawCBC} (k, (m, t \oplus m)) &=& F(k, F(k, m) \oplus (t \oplus m)) \\
  &=& F(k, t \oplus (t \oplus m)) \\
  &=& t
\end{eqnarray}

\subsection{ECBC-MAC and NMAC analysis}
\emph{Theorem: } for any $L > 0$, for every efficient $q$-query PRF adversary
$A$ attacking $F_\text{ECBC}$ or $F_{NMAC}$ there exist an efficient adversary
$B$ such that
\begin{eqnarray}
  \text{Adv}_\text{PRF}[A, F_\text{ECBC}] &\le& \text{Adv}_\text{PRP}[B, F] + 2
  \frac{q^2}{|\mathcal{X}|}\\
  \text{Adv}_\text{PRF}[A, F_\text{NMAC}] &\le& q L \text{Adv}_\text{PRF}[B, F]
  + \frac{q^2}{2 |\mathcal{X}|}
\end{eqnarray}

The bottom lines are
\begin{itemize}
\item CBC-MAC is secure as long as $q \ll |\mathcal{X}|^{1/2}$
\item NMAC is secure as long as $q \ll |\mathcal{K}|^{1/2}$
\end{itemize}

For example, suppose we want the advantag $\text{Adv}_\text{PRF}[A, F_{ECBC} \le
1/2^{32}$ then $q^2 / |\mathcal{X} \le 1 / 2^{32}$,
\begin{itemize}
\item for AES, $|\mathcal{X}| = 2^{128}$, then $q < 2^{48}$. So, after $2^{48}$
  messages we must change key.
\item for 3DES, $|\mathcal{X}| = 2^{64}$, then $q < 2^{16}$.
\end{itemize}

The security boundaries are tight. Similar to the birthday paradox, after
signing $|\mathcal{X}|^{1/2}$ messages with ECBC-MAC or $|\mathcal{K}|^{1/2}$
messages with NMAC, the probability of running into the same message/key is
significant, and the MACs becomes insecure because of the extension property.

Suppose the underlying PRF $F$ is a PRP (for example AES) then both PRFs (ECBC
and NMAC) have the following extension property:
\begin{eqnarray}
  \forall x, y, w: F_\text{BIG} (k, x) = F_\text{BIG} (k, y) &\Rightarrow&
  F_\text{BIG}(k, x||w) = F_\text{BIG}(k, y || w) \label{extension}
\end{eqnarray}

Let $F_\text{BIG}: \mathcal{K} \times \mathcal{X} \to \mathcal{Y}$ be a PRF that
has the extension property \eqref{extension}, an adversary can design a generic
attack on the derived MAC as follows:
\begin{enumerate}
  \item issue $|\mathcal{Y}|^{1/2}$ message queries for random messages in
    $\mathcal{X}$ and obtain all the pairs $(m_i, t_i)$ for $i = 1, \dotsc,
    |\mathcal{Y}|^{1/2}$.
  \item find a collision $t_u = t_v$ for $u \neq v$, which exists with high
    probability by the birthday paradox.
  \item chose some $w$ and query for $t = F_\text{BIG}(k, m_u || w)$
  \item output forgery $(m_v || w, t)$. Indeed $t = F_\text{BIG}(k, m_v || w)$.
\end{enumerate}

For better security, we can use random construction.

\subsection{Comparison}
ECBC-MAC is commonly used as an AES-based MAC
\begin{itemize}
\item CCM encryption mode (used in 802.11i)
\item NIST standard called CMAC
\end{itemize}

NMAC is not usually used with AES or 3DES because of the main reasion is the
need to change AES key on every block, and AES was not designed to perform well
when the key is changed on every block: it requires re-computing AES key
expansion. But NMAC is the basis for a popular MAC called HMAC.

\section{MAC padding}
In ECBC-MAC, we iteratively go through blocks of message to get to the final
tag. The number of blocks can be arbitrary as long as it is less than a maximum
value $L$. But what if the message length is not a multiple of block size? We
will have to pad the message with some binary string to complete the block, but
what string should we pad? Can we just pad $m$ with 0's? This is a bad idea
because given the tag for $m$ the attacker can obtain the tag for $m||0$ (which
is the same as the tag for $m$).

For security, the padding must be invertible, if $m_0 \neq m_1$ then pad($m_0$)
must be different from pad($m_1$): there should be no ambiguity.

ISO: pad with ``$100 \ldots 00$'' and add dummy block if needed. The dummy block
is important for the messages with length exactly multiple of block
size. For example, the first message $m_1$ needs 3 additional bits, and would be
padded as $m_1||100$ before computing the tage $t_1$. Now if the message $m_2 =
m_1||100$ does not have the dummy block padded at the end, its tage $t_2$ will
be identical to $t_1$.

CMAC (NIST standard): a variation of CBC-MAC where key = $(k, k_1, k_2)$. Up to
the last block, the encryption is identical to CBC. At the last block, if the
message does not complete the block, pad it with $100 \ldots 00$ and xor with
$k_1$ before complete the last step of CBC. If the message complete the last
block, then use $k_2$.
\begin{itemize}
\item no final encryption step (extension attack thwarted by last key xor)
\item no dummy block (ambiguity resolved by used of $k_1$ or $k_2$)
\end{itemize}

\section{PMAC and Carter-Wegman MAC}
So far we have seen ECBC-MAC and NMAC, but both of them are sequential. The next
natural question is can we build a parallel MAC from a small PRF?

\end{document}