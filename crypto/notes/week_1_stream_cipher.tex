\documentclass{article}
\usepackage{amsmath}
\usepackage{mathtools}

\begin{document}

Important points:
\begin{itemize}
\item One time pad has perfect secrecy
\item Any cipher that has perfect secrecy must have key space at least as large
as the message space
\item For stream cipher to be secure, the PRG must be unpredictable.
\end{itemize}
\section{One time pad}

\emph{Definition:} a cipher defined over $(\mathcal{K, M, C})$ is a pair of 
efficient algorithms $(E,D)$ where
\begin{eqnarray}
E:&& \mathcal{K} \times \mathcal{M} \to \mathcal{C} \\
D:&& \mathcal{K} \times \mathcal{C} \to \mathcal{M}
\end{eqnarray}
such that $ \forall m \in \mathcal{M}, k \in \mathcal{K}$
\begin{eqnarray}
D(k, E(k, m)) &=& m
\end{eqnarray}

For the one time pad (OTP), the cipher text is simply
\begin{eqnarray}
c &=& m \oplus k
\end{eqnarray}
Because of the property of the XOR operator, decrypting the OTP is 
\begin{eqnarray}
m &=& c \oplus k
\end{eqnarray}
The one time pad is the first example of a ``secure'' cipher: the key is 
a random bit string as long as the message
\begin{eqnarray}
\mathcal{M} = \mathcal{C} = \mathcal{K} = \lbrace 0,1 \rbrace^n
\end{eqnarray}

We can see that the OTP is very fast to encrypt and decrypt, but the key is as
long as the message. Is the OTP secure? But what is a secure cipher anyway?
For now, we will assume that the attacker can only attack on the cipher text.
Later on in the course, we will consider other possibilities. Shannon's idea:
``cipher text should reveal no information about plain text''.

\emph{Definition:} A cipher $(E, D)$ over $(\mathcal{K, M, C}$ has perfect 
secrecy if $\forall m_0, m_1 \in \mathcal{M}$ (where len($m_1$) = len($m_0$)) 
and $\forall c \in \mathcal{C}$
\begin{eqnarray}
\text{Pr}[ E(k, m_0) = c ] &=& \text{Pr} [E(k, m_1) = c]
\end{eqnarray}
where $k$ is uniform in $\mathcal{K}$ (we write $k \leftarrow^\text{R} 
\mathcal{K}$)

If a cipher satisfies the above definition, given a CT an attacker can't tell
if the message is $m_1$ or $m_0$ (for all $m_1, m_0$). Therefore, even the most
powerful adversary learns nothing about the PT from the CT. 

We can show that OTP has perfect secrecy. However, we can also show that any 
cipher that has perfect secrecy must satisfy  
\begin{eqnarray}
|\mathcal{K}| \ge |\mathcal{M}|
\end{eqnarray}
or the key has to be at least as long as the message. 

\section{Pseudo-Random Generator}

From the previous section, we have seen that OTP has perfect secrecy. However,
OTP is not very practical because the key is as long as the message. If we have
a mechanism to transfer the key securely, we can use that same mechanism to 
transfer the message itself. Stream cipher is an approach to make OTP practical.
The idea is to replace the ``random'' key by ``pseudorandom'' key. 

\emph{Definition:} A PRG is a function $G: \lbrace 0,1 \rbrace^s \to \lbrace 0,
1 \rbrace^n$ such that $n \gg s$, where $\lbrace 0,1 \rbrace^s$ is called the 
seed space. 

Using a PRG, we can expand the key to match the length of the message and then
use OTP to encrypt the message. 

Note that stream cipher cannot have perfect secrecy because the key is much 
shorter than the message. So we will need a fifferent definition of security, 
which depends on the PRG.

For the stream cipher to be secure, the PRG must be unpredictable. 

\emph{Definition:} We say that $G: \mathcal{K} \to \lbrace 0, 1 \rbrace^n$ is 
predictable if exist an efficient algorithm and $0 \le i \le n-1$ such that
\begin{eqnarray}
\text{Pr}_{k \leftarrow^R \mathcal{K}} \left[ A(G(k))|_{1, \ldots, i} = G(k)|_{i+1}
 \right]  &>&\frac{1}{2} + \epsilon
\end{eqnarray}
for non-negligible $\epsilon$ ($\epsilon > 1 / 2^{30}$)

\emph{Definition:} a PRG is unpredictable if it is not predictable: for all $i$,
there is no efficient adversary can predict bit $i+1$ for non-negligible 
$\epsilon$. 

\section{Negigible and non-negligible}

In practice: $\epsilon$ is a scalar and 
\begin{itemize}
\item non-negligible if $\epsilon \ge 1 / 2^{30}$ (likely to happen over 1GB of
data).
\item negigible if $\epsilon \le 1 / 2^{80}$ (unlikely to happen over life of 
key)
\end{itemize}

In theory: $\epsilon(\lambda)$ is a function from positive intergers to 
positive reals
\begin{itemize}
\item non-negligible if $\exists d: \epsilon(\lambda) \ge 1 / \lambda^d$ 
infinitely often ($\epsilon \ge $ 1/polynomial for many $\lambda$)
\item negigible if $\forall d, \lambda \ge \lambda_d: \epsilon(\lambda) \le
1/\lambda^d$
\end{itemize}

\emph{Definition:} We say that $G_\lambda: \mathcal{K}_\lambda \to \lbrace 0,1 
\rbrace^{n(\lambda)}$ is predictable at position $i$  if there exist a polynomial
time (in $\lambda$) algorithm $A$ such that 
\begin{eqnarray}
\text{Pr}_{k \leftarrow \mathcal{K}_\lambda} \left[ A(\lambda, G_\lambda(k)|_{1, \ldots,
i}) =  G_\lambda(k)|_{i+1} \right] > \frac{1}{2} + \epsilon(\lambda)
\end{eqnarray}
for some non-negligible function $\epsilon(\lambda)$
\section{Attack on OTP and stream cipher}

\subsection{Attack 1: two time pad is insecured}
Never use stream cipher key more than once! For example, if we use stream cipher
twice
\begin{eqnarray}
  c_1 &\leftarrow& m_1 \oplus \text{PRG}(k)\\
  c_2 &\leftarrow& m_2 \oplus \text{PRG}(k)
\end{eqnarray}
then,
\begin{eqnarray}
  c_1 \oplus c_2 &=& m_1 \oplus m_2
\end{eqnarray}
And there is enough redundancy in English and ASCII encoding that from $m_1
\oplus m_2$ we can deduce $m_1$ and $m_2$.

Real world example of this type of attack:
\begin{itemize}
\item Project Venona
\item MS-PPTP (windows NT)
\item 802.11b WEP
\item disk encryption
\end{itemize}
\subsection{Attack2: no integrity}
Modifications to ciphertext are undetected and have \emph{predictable} impact on
plaintext. 

\section{Real-world stream cipher}
\subsection{RC4 (software)}
RC4 was used in HTTPS and WEP. The weakness are
\begin{itemize}
\item bias in initial output: Probabilty of the second byte is equal to zero is
  $2/256$ 
\item Probabilty of $(0,0)$ is $1/256^2 + 1/256^3$
\item related key attacks
\end{itemize}
\subsection{CSS (hardware - badly broken)}
Scheme: linear feedback shift register (LFSR):
\begin{itemize}
\item DVD encryption (CSS): 2 LFSRs
\item GSM encryption (A5/1,2): 3 LFSRs
\item Bluetooth (E0): 4 LFSRs
\end{itemize}
All are broken.

\subsection{Modern stream cipher: eStream}
The PRG use a seed of length $s$ concatenated with non-repeating nonce to
generate random string
\begin{eqnarray}
  E(k, m; r) &=& m \oplus \text{PRG}(k; r)
\end{eqnarray}
It is essential that the pair $(k, r)$ is never used more than once.

Example of eStream: Salsa 20 (implemented in both software and hardware).
It is unknown if Salsa 20 is secured or not, but in reality there is no known
attack better than exhautive search.

\section{PRG security Definition}

Let $G: K \to \lbrace 0, 1\rbrace^n$ be a PRG. Our goal is to define what it
means that
\begin{eqnarray}
  [k \xleftarrow{R} \mathcal{K}, \text{ output } G(k) ]
\end{eqnarray}
is ``indistinghuishable'' from
\begin{eqnarray}
  [r \xleftarrow{R} \lbrace 0, 1 \rbrace^n, \text{ output } r]
\end{eqnarray}

Let us first define what is a statistical test.
\emph{Definition:} a statistical test on $\lbrace 0, 1 \rbrace^n$ is an
algorithms $A$ such that $A(x)$ output 0 or 1.

\emph{Definition:} let $G: \mathcal{K} \to \lbrace 0,1 \rbrace^n$ be a PRG and
$A$ is a statistical test on $\lbrace 0, 1\rbrace^n$. We define the advantage of
the PRG given the statistical test $A$ as
\begin{eqnarray}
  \text{Adv}_\text{PRG}[A, G] &=& \left| \text{Pr}_{k \xleftarrow{R}
      \mathcal{K}} [A(G(k)) = 1] - \text{Pr}_{r \xleftarrow{R} \lbrace 0,
      1 \rbrace^n }[A(r) = 1] \right|
\end{eqnarray}
If the advantage is close to 1, then the statistical test can distinguish $G$
from a real random number generator. If the advantage is close to zero, then $A$
cannot distinguish $G$ from a real random number generator.

\emph{Definition:} we say that $G: \mathcal{K} \to \lbrace 0, 1 \rbrace^n$ is a
secure PRG if for all ``efficient'' statistical test, the advantage of $G$ is
``negigible''.

Are there provable secure PRG? No, but we have heuristic candidates.

\emph{Theorem (Yao '82):} Let $G: K \to \lbrace 0, 1 \rbrace^n$ be a PRG, if for
all $i \in \lbrace 0, \dotsc, n-1 \rbrace$ PRG $G$ is unpredictable at position
$i$ then $G$ is a secure PRG.

In other words, if next-bit predictors cannot distinguish $G$ from random then
no statistical test can.

In general, we can have define computationally indistinghuishable as follow:

\emph{Definition:} let $P_1$ and $P_2$ be two distributions over $\lbrace 0, 1
\rbrace^n$. We say that $P_1$ and $P_2$ are computationally indistinghuishable
(denote $P_1 \approx_p P_2$) if for all efficient statistical test $A$
\begin{eqnarray}
   \left| \text{Pr}_{x \leftarrow P_1} [A(x) = 1] - \text{Pr}_{x \leftarrow P_2}
     [A(x) = 1] \right|
\end{eqnarray}
is negigible.

We can say a PRG is secure if it is computationally indistinghuishable from a
uniform distributions over $\lbrace 0, 1 \rbrace^n$.

\section{Semantic security}

\section{Stream cipher is Semantically secure}
\end{document}
 