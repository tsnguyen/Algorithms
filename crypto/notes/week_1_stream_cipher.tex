\documentclass{article}
\usepackage{amsmath}

\begin{document}

Important points:
\begin{itemize}
\item One time pad has perfect secrecy
\item Any cipher that has perfect secrecy must have key space at least as large
as the message space
\item For stream cipher to be secure, the PRG must be unpredictable.
\end{itemize}
\section{One time pad}

\emph{Definition:} a cipher defined over $(\mathcal{K, M, C})$ is a pair of 
efficient algorithms $(E,D)$ where
\begin{eqnarray}
E:&& \mathcal{K} \times \mathcal{M} \to \mathcal{C} \\
D:&& \mathcal{K} \times \mathcal{C} \to \mathcal{M}
\end{eqnarray}
such that $ \forall m \in \mathcal{M}, k \in \mathcal{K}$
\begin{eqnarray}
D(k, E(k, m)) &=& m
\end{eqnarray}

For the one time pad (OTP), the cipher text is simply
\begin{eqnarray}
c &=& m \oplus k
\end{eqnarray}
Because of the property of the XOR operator, decrypting the OTP is 
\begin{eqnarray}
m &=& c \oplus k
\end{eqnarray}
The one time pad is the first example of a ``secure'' cipher: the key is 
a random bit string as long as the message
\begin{eqnarray}
\mathcal{M} = \mathcal{C} = \mathcal{K} = \lbrace 0,1 \rbrace^n
\end{eqnarray}

We can see that the OTP is very fast to encrypt and decrypt, but the key is as
long as the message. Is the OTP secure? But what is a secure cipher anyway?
For now, we will assume that the attacker can only attack on the cipher text.
Later on in the course, we will consider other possibilities. Shannon's idea:
``cipher text should reveal no information about plain text''.

\emph{Definition:} A cipher $(E, D)$ over $(\mathcal{K, M, C}$ has perfect 
secrecy if $\forall m_0, m_1 \in \mathcal{M}$ (where len($m_1$) = len($m_0$)) 
and $\forall c \in \mathcal{C}$
\begin{eqnarray}
\text{Pr}[ E(k, m_0) = c ] &=& \text{Pr} [E(k, m_1) = c]
\end{eqnarray}
where $k$ is uniform in $\mathcal{K}$ (we write $k \leftarrow^\text{R} 
\mathcal{K}$)

If a cipher satisfies the above definition, given a CT an attacker can't tell
if the message is $m_1$ or $m_0$ (for all $m_1, m_0$). Therefore, even the most
powerful adversary learns nothing about the PT from the CT. 

We can show that OTP has perfect secrecy. However, we can also show that any 
cipher that has perfect secrecy must satisfy  
\begin{eqnarray}
|\mathcal{K}| \ge |\mathcal{M}|
\end{eqnarray}
or the key has to be at least as long as the message. 

\section{Pseudo-Random Generator}

From the previous section, we have seen that OTP has perfect secrecy. However,
OTP is not very practical because the key is as long as the message. If we have
a mechanism to transfer the key securely, we can use that same mechanism to 
transfer the message itself. Stream cipher is an approach to make OTP practical.
The idea is to replace the ``random'' key by ``pseudorandom'' key. 

\emph{Definition:} A PRG is a function $G: \lbrace 0,1 \rbrace^s \to \lbrace 0,
1 \rbrace^n$ such that $n \gg s$, where $\lbrace 0,1 \rbrace^s$ is called the 
seed space. 

Using a PRG, we can expand the key to match the length of the message and then
use OTP to encrypt the message. 

Note that stream cipher cannot have perfect secrecy because the key is much 
shorter than the message. So we will need a fifferent definition of security, 
which depends on the PRG.

For the stream cipher to be secure, the PRG must be unpredictable. 

\emph{Definition:} We say that $G: \mathcal{K} \to \lbrace 0, 1 \rbrace^n$ is 
predictable if exist an efficient algorithm and $0 \le i \le n-1$ such that
\begin{eqnarray}
\text{Pr}_{k \leftarrow^R \mathcal{K}} \left[ A(G(k))|_{1, \ldots, i} = G(k)|_{i+1}
 \right]  &>&\frac{1}{2} + \epsilon
\end{eqnarray}
for non-negligible $\epsilon$ ($\epsilon > 1 / 2^{30}$)

\emph{Definition:} a PRG is unpredictable if it is not predictable: for all $i$,
there is no efficient adversary can predict bit $i+1$ for non-negligible 
$\epsilon$. 

\section{Negigible and non-negligible}

In practice: $\epsilon$ is a scalar and 
\begin{itemize}
\item non-negligible if $\epsilon \ge 1 / 2^{30}$ (likely to happen over 1GB of
data).
\item negigible if $\epsilon \le 1 / 2^{80}$ (unlikely to happen over life of 
key)
\end{itemize}

In theory: $\epsilon(\lambda)$ is a function from positive intergers to 
positive reals
\begin{itemize}
\item non-negligible if $\exists d: \epsilon(\lambda) \ge 1 / \lambda^d$ 
infinitely often ($\epsilon \ge $ 1/polynomial for many $\lambda$)
\item negigible if $\forall d, \lambda \ge \lambda_d: \epsilon(\lambda) \le
1/\lambda^d$
\end{itemize}

\emph{Definition:} We say that $G_\lambda: \mathcal{K}_\lambda \to \lbrace 0,1 
\rbrace^{n(\lambda)}$ is predictable at position $i$  if there exist a polynomial
time (in $\lambda$) algorithm $A$ such that 
\begin{eqnarray}
\text{Pr}_{k \leftarrow \mathcal{K}_\lambda} \left[ A(\lambda, G_\lambda(k)|_{1, \ldots,
i}) =  G_\lambda(k)|_{i+1} \right] > \frac{1}{2} + \epsilon(\lambda)
\end{eqnarray}
for some non-negligible function $\epsilon(\lambda)$
\end{document}
