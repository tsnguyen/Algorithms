\documentclass{article}
\usepackage{amsmath}
\usepackage{mathtools}

\title{Block Cipher}

\begin{document}
\maketitle

\section{What is block cipher?}
Block cipher is the cryto work horse, which encrypt $n$ bits block of PT using
$k$ bits key to get $n$ bits block of CT. The canonical examples for block
ciphers are:
\begin{itemize}
\item 3DES: $n = 64$, $k = 168$
\item AES: $n = 128$, $k = 128, 192, 256$
\end{itemize}

Block cipher is built by iteration: the key $k$ is expanded into $n$ keys using
key expansion. Then the PT $m$ goes through $n$ rould of round function $R(k_i,
m)$. For 3DES, the number of round functions is 48, for AES-128 the number of
round funciton is 10. Because of the iteration, block cipher is generally slower
than stream cipher.

To study block cipher at the abstract level, let us define pseudo random
permutaiton and pseudo random function. 

\emph{Definition: } pseudo random function (PRF) is defined over $(\mathcal{K,
  X, Y})$ :
\begin{eqnarray}
  F: \mathcal{K} \times \mathcal{X} \to \mathcal {Y}
\end{eqnarray}
such that exist an ``efficien'' algorithm to evaluate $F(k, x)$.

\emph{Definition: } pseudo random permutaiton (PRP) is defined over
$(\mathcal{K, X})$:
\begin{eqnarray}
  E: \mathcal{K} \times \mathcal{X} \to \mathcal{X}
\end{eqnarray}
such that
\begin{enumerate}
\item exist an ``efficient'' deterministic algorithm to evaluate $e(K,X)$
\item the function $E(k, \cdot) $ is one to one
\item exist an ``efficien'' inversion algorithm $D(k, m)$
\end{enumerate}

Functionally, any PRP is also a PRF: a PRP is a PRF where $\mathcal{X} =
\mathcal{Y}$ and efficiently invertible.

\subsection{Secure PRFs}

Let $F: \mathcal{K} \times \mathcal{X} \to \mathcal{Y}$ be a PRF. We can define
two sets of functions:
\begin{itemize}
  \item Funs$[\mathcal{X, Y}]$: is the set of all functions from $\mathcal{X}$
    to $\mathcal{Y}$.
  \item $S_F$ is the set of function $F(k, \cdot)$ such that $k \in
    \mathcal{K}$, which is a subset of Funs$[\mathcal{X, Y}]$.
\end{itemize}

Intuition: a PRF is secure if a random function in Funs$[\mathcal{X,Y}]$ is
indistinguishable from a random function in $S_F$. What does this mean? Suppose
an adversary gives us a value $x \in \mathcal{X}$, and then we return with
either $f(x)$ or $F(k,x)$. $F$ is a secure PRF if the adversary cannot tell if
the returned value was from $f(x)$ or $F(k, x)$.

\subsection{Secure PRPs (or secure block ciper)}

Let $E: \mathcal{K} \times \mathcal{X} \to \mathcal{Y}$ be a PRP. We can define
two sets of functions as
\begin{itemize}
  \item Perms$[\mathcal{X}]$ is the set of all one to one function fom $X$ to
    $Y$
  \item $S_F$ is the set of all function $E(k, \cdot)$ such that $k \in
    \mathcal{K}$. $S_F$ is a subset of Perms$[\mathcal{X}]$. 
\end{itemize}

Intuition: a PRP is secure if a random function in Perms$[\mathcal{X}]$ is
indistinguishable from a random function in $S_F$.

\subsection{An easy application: from PRF to PRG}

Let $F: \mathcal{K} \times \lbrace 0, 1 \rbrace^n \to \lbrace 0, 1 \rbrace^n$ be
a secure PRF. Then the following $G: \mathcal{K} \to \lbrace 0, 1 \rbrace^{n t}$
is a secure PRG.
\begin{eqnarray}
  G(k) &=& F(k, 0) || F(k,1) || \dotsb || F(k, t-1)
\end{eqnarray}
This PRG is secure and parallelizable.

\section{The data encryption standard (DES)}

Lucifer was designed by Horst Feistel in early 1970s, and later  adopted by the
National Bureau of Standard in  1976 as DES. In 1997, DES was broken by
exhautive search. In 2000, NIST adopted Rijndael as AES to replace DES.

DES is widely deployed in banking and commerce.

The core idea of DES is the Feistel network: suppose we are given a set of
arbitrary functions $f_1, \dotsc, f_d: \lbrace 0, 1 \rbrace^n \to \lbrace 0, 1
\rbrace^n$, we would want to build an invertible function $F: \lbrace 0, 1
\rbrace^{2 n} \to \lbrace 0, 1 \rbrace^{2n}$. The Feistel network is constructed
as follow
\begin{eqnarray}
  \left\lbrace \begin{array}{ccl}
      R_i & = & f_i (R_{i-1} ) \oplus L_{i-1}\\
      L_i & = & R_{i-1}
    \end{array}\right. 
\end{eqnarray}
where $L_i$ and $R_i$ are the first and the second halves of the $2n$ bits at
layer $i$.

The inverse is constructed as follow
\begin{eqnarray}
  \left\lbrace \begin{array}{ccl}
      R_{i-1} &=& L_i \\
      L_{i-1} &=& f_i(L_i) \oplus R_i
    \end{array}\right.
\end{eqnarray}
So, the inversion is basically the same, with the functions $f_1, \dotsc, f_d$
applied in reversed order.

The Feistel network gives a general method for building invertible functions
(block cipher) from arbitrary functions. This construction is used in many block
ciphers but not in AES.

\emph{Theorem (Luby - Rackoff '85):} given $f: \mathcal{K} \times \lbrace 0, 1
\rbrace^n \to \lbrace 0, 1 \rbrace^n$ is a secure PRF, then a 3 round Feistel
$F: \mathcal{K}^3 \times \lbrace 0, 1 \rbrace^{2 n} \to \lbrace 0, 1
\rbrace^{2n}$ is a secure PRP. Note that it is essential to have 3 independent
keys. 

DES is a 16 round Feistel network where $f_i(x) = F(k_i, x) $ with $k_i$
obtained from the key expansion. 
\end{document}